%Preamble:
\documentclass[10pt,twocolumn,letterpaper]{article} %possibilities: book, report, etc
%change the fontsize to 12pt, etc if desired
%Also possible \documentclass[10pt,twocolumn]{article}
%Also possible: \documentclass{CSUthesis} (our template)

%Change margins: Usually taken care of by style file from journal, etc.
\textwidth=6.85in
\oddsidemargin=-0.5in
\textheight=9in
\topmargin=-0.75in

\usepackage{url,epsfig,graphics,float}
\usepackage{color}
\author{Hanz Richter \\ Dept. of Mechanical Engineering} %use \\ to force a line break
\date{\today} %for no date use \date{\empty}
\title{Introduction to \LaTeX} %Latex has its own command, and it's case-sensitive

\begin{document}
%to suppress page numbering you can use \pagestyle{\empty} (whole doc)
\maketitle
\tableofcontents
\listoffigures
\listoftables
\section{Origins of \LaTeX}
\LaTeX started as \TeX, a \emph{typesetting} %use \emph for italics
program designed by Donald Knuth, a Stanford computer scientist who became frustrated with the poor quality of mathematical equations typed using a typewriter. %note that line breaks in the source tex file will be ignored unless you use \\ or leave a blank line, like next.

\noindent For information about him, see here:\\ %use \noindent if you want to avoid indentation at the beginning of the paragraph
\url{www-cs.faculty.standford.edu/~uno/}\\ %don't forget to include the url package if including hyperlinks
\subsection{Differences between Word and \LaTeX}\label{ssdiff} %if you want to refer to a section later in the document, %include a tag of your own choosing (``ssdiff'' was chosen here).
The main differences between \textbf{typesetting} programs like \LaTeX and conventional word processors are:
\begin{enumerate}
\item Word processors are \emph{WYSIWYG} (``what you see is what you get'', you see the results as you go)
\item Unfortunately, sometimes you don't like what you see! (think Microsoft's Equation Editor)
\item \LaTeX requires a source text file (this file) called the ``tex file'' which must be compiled in order to see the results, much like C programming.
\item With typesetting, you forget about the formatting and concentrate on the contents. No need to use tabs or worry about the placement of figures and tables, it's automatic and dictated by the declared style.
\item \textbf{Version-free:} The source tex file is just a text file, it never becomes obsolete due to version problems (unlike MS Word). You can type the source file in Notepad or any text editor.
\item The compiled document is \emph{device-independent}. A dvi file can be read in a Mac, PC or in UNIX/Linux.
\item \LaTeX is \textcolor{red}{free}. There are thousands of styles and packages for creating math formulas, chemical equations, music notation, etc. Also, there are plenty or language options (the babel package).
\item There are WYSIWYG packages that work with \LaTeX: Scientific Workplace, LyX and others, but they are not as flexible as tex source coding and some are not free.
\end{enumerate}
\section{The Basic Compilation Process}
As we mentioned in Section~\ref{ssdiff}, the source file must be compiled. The basic process is as follows:
\begin{enumerate}
\item \underline{Run \LaTeX on the tex source to generate a dvi file}: You can use a plain text editor, but tex-oriented editors are really convenient: WinEdt, WinTeX, TeXnic Center, TeXWorks, etc. Most must be purchased to avoid annoying popups. Kile works very well in Linux platforms (highly recommended).
\item \underline{Preview the results}: You can do this in most tex-oriented editors.
\item \underline{Export to pdf}:  You can do this in most tex-oriented editors.
\item The above 3 steps can also be run from the command line in Windows and Unix systems.
\item Often, \LaTeX has to be compiled twice to resolve cross-referencing of labels.
\end{enumerate}
\section{Typesetting Formulas}
This is the main reason to use \LaTeX. There are three types of equation entries. Numbered equations:
\begin{equation}
\dot{x}=\frac{dx}{dt} \label{derivative}
\end{equation}
Unnumbered equations (formatting may not be very good for large symbols):
\[ \int_0^{\infty} f(s)ds \]
Inline equations appear in the middle of the text: $x^2+2x+1=\sin(t)$. Numbered equations (like Eq.~\ref{derivative}) may be forced to have no number:
\begin{equation}
A=\left [ %use \left [ instead of just [ to adjust the size of the bracket automatically
\begin{array}{llrr} %denotes left or right alignment for each column
1 & 2 & 1.34 & 1.789 \\ %separate columns with & , change lines with \\
0 & -1 & 2 & 4.389 \\
z & y & a-b & 0  %no \\ needed for the last line
\end{array} \right ] \nonumber
\end{equation}
Multiline equations are also possible:
\begin{eqnarray*} %use * to avoid numbering of each equation 
\dot{z}_1 &=& z_2 \label{stateq1} \\
\dot{z}_2 &=& M^{-1}(Ku-B-Cz_2-F_f-J_e^TF_e-g(q))
\end{eqnarray*}
A second sample document will be made available that has many formula typesetting examples.
\section{Tables}
The format for tables is similar to the one used for matrices:
\begin{table}
\begin{center}
\begin{tabular}{|l|r|r|} % the vertical bars are used to draw vertical grid lines on the table
\hline %\hline is used for horizontal grid lines, and it already has the line break
Name & Symbol & Units \\
\hline 
Pressure & $p$ & Pa \\ %note that any math must use the inline form, with $ signs.
\hline 
Kinetic Energy & $\frac{1}{2}mv^2$ & J \\
\hline
Cost & $C$ & \$  \\ %use \$ to display a dollar sign without error
\hline
\end{tabular} \caption{Definitions} \label{mytable} %caption and label must be defined in that order
\end{center}
\end{table}
Tables will tend to go to the top of the page. We can force other locations using the float package:
\begin{table}[hb] %HB means ``here bottom''
\begin{center}
\begin{tabular}{|l|r|r|} % the vertical bars are used to draw vertical grid lines on the table
\hline %\hline is used for horizontal grid lines, and it already has the line break
Name & Symbol & Units \\
\hline
Pressure & $p$ & Pa \\ %note that any math must use the inline form, with $ signs.
\hline
Kinetic Energy & $\frac{1}{2}mv^2$ & J \\
\hline
Cost & $C$ & \$  \\ %use \$ to display a dollar sign without error
\hline
\end{tabular} \caption{Repeat Table} \label{myothertable} %caption and label must be defined in that order
\end{center}
\end{table}
As we see, Table~\ref{myothertable} tries to appear closest to where we code it, or lower on the page (here, bottom). 
\section{Including Figures}
You can include external figure files or code that creates drawings. You can embed \emph{encapsulated PostScript} (eps) files. Many graphics programs (most importantly, Matlab) can export a figure to eps:
\begin{figure}
\centering
\includegraphics[width=2in]{robot_coords.eps} \caption{Denavit-Hartenberg Coordinate Frame Assignments}
\label{fig_coords} %caption and label must be defined in that order
\end{figure}
Figure~\ref{fig_coords} was created using IPE, a nice graphics editor that can include \LaTeX coding in the diagrams. IPE is free  and easy to use. Figure~\ref{figmlab} was generated in Matlab.
\begin{figure}
\centering
\includegraphics[width=3in,height=2in,angle=36.7]{force.eps} \caption{Vertical Ground Force}
\label{figmlab}
\end{figure}
\section{Bibliographies}
There are two ways to include bibliographical citations with automatic numbering:
\begin{enumerate}
 \item [a] Using a bibliography list embedded in the tex source file (OK for short lists) %use [a] to force this symbol instead of 1,2..
 \item [b] Using a separate BibTeX file (text file with special formatting, extension .bib). This is the recommended method, works well with long bibliography databases or multiple databases. Many scientific paper catalogs can export article citations to BibTeX. 
\end{enumerate}
As an example, the bibliography shown at the end of this document was created using separate files called {\tt myrefs1.bib} and {\tt myrefs2.bib}. %use {\tt } to produce a monospaced (computer-like) font. 
The procedure to compile bibliographies is as follows:
\begin{enumerate}
 \item Run \LaTeX
 \item Run BibTeX (on the tex source file)
 \item Run \LaTeX again to resolve cross references.
\end{enumerate}
\subsection{Sample text with citations}
A good design book is Slocum~\cite{slocum}.


Constraint enforcement via model-predictive control, although effective, is widely-regarded as a computationally-intensive process, although efforts have been made to reduce its  complexity~\cite{bempoTAC02,bempoaut02,AIAA-MPC}.


\bibliographystyle{plain}
\bibliography{myrefs1,myrefs2}
\end{document}
